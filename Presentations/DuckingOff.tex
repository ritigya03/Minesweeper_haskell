\documentclass{beamer}
\usetheme{CambridgeUS}
\usecolortheme{crane}
\usefonttheme{serif}

\usebackgroundtemplate%
{%
    \includegraphics[width=\paperwidth,height=\paperheight]{duckies.jpg}%
}


% Title page details: 
\title{Ducking Off} 
\subtitle{Implementing Minesweeper in Haskell}
\author{Kriti Chaturvedi \and Mariam Eqbal \and Ritigya Gupta}
\date{\today}

\titlegraphic{\includegraphics[width=.5\textwidth,height=.5\textheight]{duckie.png}}

\begin{document}

% Title page frame
\begin{frame}
    \titlepage 
\end{frame}

\section{Objective}
%Objective Frame
\begin{frame}{Objective}
    \begin{enumerate}
        \item To learn and establish a basic proficiency in Haskell.
        \item To familiarize ourselves with GUI libraries in Haskell.
        \item To develop an attitude to learn things in limited time constraints.
    \end{enumerate}
\end{frame}

\section{Some fun modifications}
%Objective Frame
\begin{frame}{Some fun modifications}
    \begin{enumerate}
        \item Unopened tiles will be respresented by lilypads. 
        \item Numbered tiles will be represented by water.
        \item Mines will be represented by sharks.
        \item Safe areas will be represented by ducks. 
        \item Flags will be represented by lilies.
    \end{enumerate}
\end{frame}

\section{Why Haskell?}
%Why Haskell Frame
\begin{frame}{Why Haskell?}
    \begin{enumerate}
        \item We just want to learn it
        \item High level abstractions
        \item Lazy Evaluation
        \item Purely functional language
    \end{enumerate}
    \end{frame}

\section{Implementation}
%Plan Frame
\begin{frame}{Implementation}
    \begin{enumerate}
        \item Using System.Random module for placing the mines (sharks) randomly for board generation.
        \item Numbering the tiles on the basis of the sharks around it.
        \item Use Gtk2Hs for making a GUI in Haskell.
    \end{enumerate}
    \end{frame}

\section{Division of Work}
%Work division Frame
\begin{frame}{Division of work}
    \begin{enumerate}
        \item Mariam - Basic logic of the Minesweeper board and implementation in the terminal (without GUI)
        \item Ritigya - Making an attractive Haskell GUI
        \item Kriti - Testing and debugging 
    \end{enumerate}
    \end{frame}

\section{Timeline}
%Timeline Frame
\begin{frame}{Timeline}
    \begin{enumerate}
        \item WEEK 1: \begin{itemize}
            \item Make test cases (3 boards for testing at different stages)
            \item Make the board from scratch
            \item Build GUI
            \item Integrate all components
          \end{itemize}
        
        \item WEEK 2: \begin{itemize}
            \item Days 1 - 3: Finalising the code, structuring, ordering and completing code.
            \item Days 4 - 7: Final testing and debugging
          \end{itemize}
    \end{enumerate}
    \end{frame}

\section{Potential Challenges}
%Potential Challenges Frame
\begin{frame}{Potential Challenges}
    \begin{enumerate}
        \item Coordination across regions.
        \item Creating a UI in Haskell can be difficult, especially a graphical UI.
    \end{enumerate}
    \end{frame}

\section{Questions}
%Potential Challenges Frame
\begin{frame}{ }
    \begin{center}
        \LARGE
   This is Team Project Puff Girls signing off, \\ 
   We're open to questions if any.
    \end{center}
\end{frame}
\end{document}