\documentclass{beamer}
\usetheme{CambridgeUS}
\usecolortheme{crane}
\usefonttheme{serif}

\usebackgroundtemplate%
{%
    \includegraphics[width=\paperwidth,height=\paperheight]{duckies.jpg}%
}


\title{Progress Report} 
\subtitle{Ducking Off : Implementing Minesweeper in Haskell}
\author{Kriti Chaturvedi \and Mariam Eqbal \and Ritigya Gupta}
\date{\today}

\titlegraphic{\includegraphics[width=.5\textwidth,height=.5\textheight]{duckie.png}}

\begin{document}


\begin{frame}
    \titlepage 
\end{frame}

\section{Objective}
%Objective Frame
\begin{frame}{Objective}
    \begin{enumerate}
        \item To learn and establish a basic proficiency in Haskell.
        \item To familiarize ourselves with GUI libraries in Haskell.
        \item To develop an attitude to learn things in limited time constraints.
    \end{enumerate}
\end{frame}

\section{Why (not) Haskell?}
%Objective Frame
\begin{frame}{Why (not) Haskell?}
    \begin{enumerate}
        \item Haskell's immutability is not something suitable for maintaining game states.
        \item GUI programming was awkaward to write in a purely functional language
    \end{enumerate}
\end{frame}

\section{Learnings}

\begin{frame}{Explicit Learnings}
    \begin{enumerate}
        \item Proficiency in Haskell
        \item Understanding GTK
        \item Handling containers in Haskell
    \end{enumerate}
\end{frame}

\begin{frame}{Implicit Learnings}
    \begin{enumerate}
        \item Learning to use git well
        \item Learning to use the terminal effectively
        \item Importance of regularly contacting and coordinating with team mates
    \end{enumerate}
\end{frame}

\section{Challenges}

\begin{frame}{Challenges}
    \begin{enumerate}
        \item Working with GTK with not much information available easily
    \end{enumerate}
\end{frame}

\section{Thank You}

\begin{frame}{ }
    \begin{center}
        \LARGE
        Thank You
    \end{center}
\end{frame}
\end{document}